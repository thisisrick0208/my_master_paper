% !TEX root = ../main.tex

\chapter{结论与展望}

本章节是对前几个章节研究和讨论内容的总结。

\section{结论}

本文主要对如何设计一种在CIS芯片上的神经网络加速器进行了研究和讨论。
首先调研和归纳了CIS芯片的发展历程、人工智能算法的发展历程、人工智能运算硬件的发展历程,以及方案中涉及的同类产品的设计思路和算法逻辑。
提出了,在CIS芯片的图像数据信号通路上,放置神经网络加速器的设计思想。
并且,提出了使用脉动阵列的思想代替了普通的神经网络运算矩阵,优化了运算性能。
文中详细描述了所有部件的功能和行为逻辑。

本文研究的成果是一种低功耗、低延迟、高处理速度、专用于设备端做推理运算的神经网络加速方案。
这种方案的性能表现大幅优于CIS芯片通过标准接口(CSI 2.0)将图像数据输送到主机后再进行运算的传统方案。
同时,基于SystemC进行建模,实现了高级语言下芯片的顶层控制逻辑和算法。



\section{展望}
基于国际数据公司(IDC)的研究报告,ASIC、FPGA、NPU等其他非GPU芯片也在各个行业和领域被越来越多地采用,整体市场份额接近10\%,预计到2025年其占比将超过20\%。
报告中还指出,2020年中国数据中心用于推理的芯片和市场份额已经超过50\%,预计到2025年,用于推理的工作负载的芯片将达到60.8\%。
% 2021-2022中国人工智能计算力发展评估报告
根据上述数据,无论是作为ASIC和NPU的半导体器件类型,还是专用于推理的产品用途,本课题的研究方向都是整个市场的大势所趋。
另外,由于设计芯片的工作量是非常巨大的,涉及到的领域也非常之多。
本课题仅仅在功能需求、顶层控制逻辑及算法设计、高级语言建模这几个方面进行了研究。
我相信,如果对这项课题进行更深入一步的探究,包括数字逻辑设计、形式验证、版图设计等等后续的设计和开发,甚至于最后流片,那么该芯片一定会成为一款非常有意义的带有神经网络加速器的CIS芯片。
它的意义在于充分实践了边缘计算的思想,并且实践了“软件定义硬件”的新概念。

