% !TEX root = ../main.tex

\chapter{结果与报告}

\section{测试用例}

\subsection{算法的可行性}



\subsection{空间占用和并行性}



\section{报告}

按照教务处的要求,参考文献外观应符合国标 GB/T 7714 的要求。模版使用 \BibLaTeX\
配合 \pkg{biblatex-gb7714-2015} 样式包
\footnote{\url{https://www.ctan.org/pkg/biblatex-gb7714-2015}}
控制参考文献的输出样式,后端采用 \pkg{biber} 管理文献。

请注意 \pkg{biblatex-gb7714-2015} 宏包 2016 年 9 月才加入 CTAN,如果你使用的
\TeX\ 系统版本较旧,可能没有包含 \pkg{biblatex-gb7714-2015} 宏包,需要手动安装。
\BibLaTeX\ 与 \pkg{biblatex-gb7714-2015} 目前在活跃地更新,为避免一些兼容性问
题,推荐使用较新的版本。

正文中引用参考文献时,使用 \verb|\cite{key1,key2,key3...}| 可以产生“上标引用的
参考文献”,如 \cite{Meta_CN,chen2007act,DPMG}。使用
\verb|\parencite{key1,key2,key3...}| 则可以产生水平引用的参考文献,例如
\parencite{JohnD,zhubajie,IEEE-1363}。请看下面的例子,将会穿插使用水平的和上标的
参考文献:关于书的\parencite{Meta_CN,JohnD,IEEE-1363},关于期刊的
\cite{chen2007act,chen2007ewi},会议论文 \parencite{DPMG,kocher99,cnproceed},硕
士学位论文\parencite{zhubajie,metamori2004},博士学位论文
\cite{shaheshang,FistSystem01,bai2008},标准文件 \parencite{IEEE-1363},技术报告
\cite{NPB2},电子文献 \parencite{xiaoyu2001, CHRISTINE1998},用户手册
\parencite{RManual}。

可以使用 \verb|\nocite{key1,key2,key3...}| 将参考文献条目加入到文献表中但不在正
文中引用。使用 \verb|\nocite{*}| 可以将参考文献数据库中的所有条目加入到文献表
中。
