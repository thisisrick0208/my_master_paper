% !TEX root = ../main.tex

\begin{summary}
这里是全文总结内容。

本课题阐述了一种CIS芯片上的深度学习神经网络加速器的设计方案,并建模进行了算法级的验证。
%总结一下设计
该设计利用了图像传感器中流水线式的数据通路的特性,将神经网络加速器设置在数据通路中。
这样的设计避免了运算时对低速存储设备(例如DRAM)的读写,来提升神经网络算法的运算速度。
同时,采用了脉动阵列来做神经网络算法中的乘加运算。
巧合的是,脉动阵列的特性也与图像传感器的数据通路特性一致。  
因此,无论从节省存储设备空间的角度,还是从并行执行算法节省时间的角度来说,这种设计都是优于CIS芯片外进行图像识别的方案的。  

近年,“感知时代”的概念兴起。传感器与人工智能的结合正在给人类带来新的技术革新。
同时,随着摩尔定律的失效,体系结构的变革带来了“异构计算”。
而本课题的研究内容正是“感知”加上“异构”的王炸组合。
AI芯片初创公司Groq提出了“软件定义硬件”。
这种说法在业界看来虽然为时尚早,但我们也可以从中看到,在追求极致性能的产品上,被开发的远不只是软件,硬件也是需要重新定义的。


\end{summary}
